%\iffalse
\let\negmedspace\undefined
\let\negthickspace\undefined
\documentclass[journal,12pt,onecolumn]{IEEEtran}
\usepackage{cite}
\usepackage{amsmath,amssymb,amsfonts,amsthm}
\usepackage{algorithmic}
\usepackage{graphicx}
\usepackage{textcomp}
\usepackage{xcolor}
\usepackage{txfonts}
\usepackage{listings}
\usepackage{enumitem}
\usepackage{mathtools}
\usepackage{gensymb}
\usepackage{comment}
\usepackage[breaklinks=true]{hyperref}
\usepackage{tkz-euclide} 
\usepackage{listings}
\usepackage{gvv}                                        
\def\inputGnumericTable{}                                 
\usepackage[latin1]{inputenc}                                
\usepackage{color}                                            
\usepackage{array}                                            
\usepackage{longtable}                                       
\usepackage{calc}                                             
\usepackage{multirow}                                         
\usepackage{hhline}
\usepackage{circuitikz}                        
\usepackage{ifthen}                                           
\usepackage{lscape}

\newtheorem{theorem}{Theorem}[section]
\newtheorem{problem}{Problem}
\newtheorem{proposition}{Proposition}[section]
\newtheorem{lemma}{Lemma}[section]
\newtheorem{corollary}[theorem]{Corollary}
\newtheorem{example}{Example}[section]
\newtheorem{definition}[problem]{Definition}
\newcommand{\BEQA}{\begin{eqnarray}}
 \newcommand{\EEQA}{\end{eqnarray}}
\newcommand{\define}{\stackrel{\triangle}{=}}
\theoremstyle{remark}
\newtheorem{rem}{Remark}
\begin{document}
 \bibliographystyle{IEEEtran}
 \vspace{3cm}
 \title{\textbf{EE 34}}
 \author{EE23BTECH11048-Ponugumati Venkata Chanakya$^{*}$% <-this % stops a space
 }
 \maketitle

 \bigskip
 \renewcommand{\thefigure}{\theenumi}
 \renewcommand{\thetable}{\theenumi}
 \textbf{QUESTION:}
In the circuit below,operational amplifier is ideal .If $V_1$ is $10$mV and $V_2$ is $50$mV the output voltage ($V_{out}$) is\\  
\begin{center}
    \begin{circuitikz}

% Op-amp
\draw
(0,0) node[op amp, yscale=1] (opamp) {}
(opamp.-) -- ++(-0.5,0)node[circ] {}to[R, european, l=$10k\ohm$] ++(-2,0)node[anchor=east] {$V_1$} node[ocirc] {}  % Non-inverting input
(opamp.+) -- ++(-0.5,0) to[R, european, l=$100k\ohm$] ++(0,-2) node[ground] {} % Inverting input
(opamp.+) -- ++(-0.5,0)node[circ] {} to[R, european, l=$10k\ohm$] ++(-2,0)node[anchor=east] {$V_2$}node[ocirc] {} % Feedback resistor
(opamp.out) -- ++(0.5,0) node[anchor=west] {$V_{out}$} node[ocirc] {}% Output
(opamp.-) -- ++(-0.5,0) -- ++(0,1)to[R, european, l=$100k\ohm$] ++(2.9,0) -- (opamp.out)node[circ] {} 
;
\draw (-4,-2.5) -- ++(6,0);
\end{circuitikz}
\end{center}
 \begin{enumerate}
     \item $100$mV
     \item $400$mV
     \item $500$mV
     \item $600$mV
 \end{enumerate}
\hfill{GATE 2019 EE}\\
\solution
Let $V_3$,$V_4$ be values of voltages negative and positive terminal respectively\\
For an ideal operational amplifier $V_3=V_4$\\
\begin{align*}
    V_4 &= 0+(V_2-0) \brak{\frac{R_1}{R_1+R_2}}\\
    V_4 &= \frac{R_1 V_2}{R_1+R_2}\\
    V_3 &= V_{out}+(V_1-V_{out})\brak{\frac{R_1}{R_1+R_2}}\\
    V_3 &= \frac{10V_1-V_{out}}{11}\\
   \implies   \frac{R_1 V_1-V_{out}R_2}{R_1+R_2} &= \frac{R_1 v_2}{R_1+R_2}\\
   V_{out} &=(V_2-V_1)\frac{R_1}{R_2}\\
   V_{out} &= (50-10)\frac{100}{10}\\
          &= 400mV
\end{align*}
 
 \end{document}
